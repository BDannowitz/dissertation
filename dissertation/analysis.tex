\chapter{Analysis}

\red{CHAPTER STATUS: PLANNING/OUTLINE}

\section{Particle Identification}

\section{Data Selection}

\subsection{Spill Level Cuts}

\subsubsection{Duty Factor}

The beam structure is delivered in \emph{bunches} 
of protons separated by 18.9ns. These bunches are better
known as \emph{RF buckets}, as they are synchronized with the 18.9ns cycle of the Fermilab RF clock. Ideally, each of these RF buckets deliver a steady number of protons per bucket.

In practice however, this is not the case, and not all the RF buckets delivered are occupied within a given spill. The fraction of buckets occupied is known as the accelerator Duty Factor, \emph{DF}, and can be measured using a reference beam counter as

\begin{eqnarray} DF = \frac{\langle I \rangle ^2}{\langle I^2 \rangle} &
with &  I = \sum_{N_trig} N_{X2T}
\end{eqnarray}

where $N_X2T$ is the number of hits in the X2T 

\subsection{Dimuon Level Cuts}

\subsection{Track Level Cuta}

\subsection{Other Cuts}

\section{Dimuon Yields}


\subsection{Binning of Data}

The $x_2$ bins for this EMC ratio measurement were chosen such that each
bin in $x_2$ has similar levels of statistics. 

Concurrent with this analysis, studies of $\bar{d}(x_2)/\bar{u}(x_2)$ and 
parton energy loss are conducted. Due to the nearly
identical source of signal across these studies (good Drell-Yan target
dimuons), a consistent selection of kinematic binning maintains a 
certain continuity among analyses. The $x_2$ binning chosen can be found in Table \ref{tab:x2bins}

\begin{table}
	\centering
	\setlength\tabcolsep{4pt}
\begin{minipage}{0.48\textwidth}
	\centering
	\begin{tabular}{ll}
		\toprule
		Bin\# & $x_2$ Range\\
		\midrule
		0 & (0.08, 0.14] \\
		1 & (0.14, 0.16] \\
		2 & (0.16, 0.18] \\
		3 & (0.18, 0.21] \\
		4 & (0.21, 0.25] \\
		5 & (0.25, 0.31] \\
		6 & (0.31, 0.53] \\
		\bottomrule
	\end{tabular}
	\caption{$x_2$ bin ranges}
	\label{tab:x2bins} 
\end{minipage}%
\hfill
\begin{minipage}{0.48\textwidth}
	\centering
	\begin{tabular}{ll}
		\toprule
		target & yield \\ 
		\midrule
		None   &  104 \\
		Empty  &  84 \\
		LH2    &  3138 \\
		LD2    &  3472 \\
		C      &  1721 \\ 
		Fe     &  1370 \\
		W      &  1553 \\
		\bottomrule
	\end{tabular}
	\caption{Raw dimuon yields for Roadset 57} 
	\label{tab:targyields} 
\end{minipage}
\end{table}

\subsection{Raw Yields}

The total number of target Drell-Yan events recorded for each target can be found in Table
\ref{tab:targyields}. The number of events broken down into 

\section{Rate Dependence and Combinatorial Background Correction}



\section{Empty/None Target Background Subtraction}



\section{Dimuon Yield Ratios}



\section{$ld_2$ Contamination Correction}



\section{Isoscalar Corrections for $^{183}W$ and $^{56}Fe$}


