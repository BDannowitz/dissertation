
\chapter{Conclusions}

SeaQuest is a fixed-target, \unit[120]{GeV} proton beam experiment at Fermilab that seeks to analyze the rare Drell-Yan process in pA collisions. With this beam energy and high instantaneous luminosities, SeaQuest has the potential to explore the high-$x_2$ kinematic range of the Drell-Yan process. This kinematic space has yet to be fully explored with Drell-Yan due to the depletion of anti-quarks at this high of $x_2$ and the rarity of the Drell-Yan process. In studying Drell-Yan data with its forward spectrometer, SeaQuest is able to isolate and define features of the sea quark distribution in the nucleon.

While a main focus is placed on studying deuterium to hydrogen and extracting $\dbar(x)/\ubar(x)$, the inclusion of heavy nuclear targets in the target rotation enable a study of Drell-Yan cross section ratios for heavy nuclei to deuterium. It has long been established that, in DIS experiments, quark distributions become modified in a nuclear medium. By studying this nuclear DY cross section ratio, we gain a measurement of $\sim\ubar_A(x)/\ubar_D(x)$, which lends a direct probe of the modification of the quark sea in the nuclear medium.

Presented in this work is the first dedicated attempt at extracting this ratio, $R^{A/D}_{DY}$ from existing SeaQuest data. Many challenges need be addressed in order to arrive at such a measurement, namely the handling of the rate dependence. Outlined here have been the first procedures and attempts at correcting for the rate dependence in such a way that the ratio measurements exhibit minimal rate dependence. While the correction procedure is kinematic dependent and leaves much to be desired, it is encouraging to have made progress on an analytical roadblock that has proven to be problematic. It is the opinion of the author that with further studies of the principles and methods described in this thesis with much larger quantities of data, a correction may soon be devised that utilize \emph{only} spectrometer variables and not kinematics (i.e. efficiencies based on a single track's trigger road and that event's intensity only).

The results presented give the first peek into what is certain to be an impact measurement. Already, with very limited statistics, SeaQuest is surpassing previous measurements (E-772) in precision and range. In the regions of measurement overlap, the preliminary results seems to indicate that the signal measured by SeaQuest is largely consistent with what has been seen in the E-772 result.  As the results currently stand, there appears to (still) be little nuclear A-dependence of the Drell-Yan cross section ratio, both integrated and versus $x_2$. This implies that perhaps the nuclear modification that leads to the DIS phenomenon known as the EMC effect is exclusively an effect of the valence quarks and does not originate from the quark sea.

As a the rate dependence correction becomes more refined, as large swaths of data not analyzed here are included, and as the tracking and analyitical cuts continue to be tuned, the author is confident that the total yield of SeaQuest will provide a truly precise window into the behavior of sea quarks at high $x_2$. Additional improvements would be to precisely analyze the compositions of the liquid deuterium target used, thereby completely eliminating one of the largest sources of systematic uncertainties.

In addition to the execution of this analytical measurement, some key contributions to the operation and performance of the experiment have been outlined. This includes the testing, prototyping, manufacturing, calibration, and installation of new photomultiplier tubes bases that are more capable of handling the high instantaneous rates that at times occur at SeaQuest. Also discussed is the management and curation of the experiment's raw, processed, and tracked data on a highly-available and queryable SQL database. This platform has allowed users to perform their analysis quickly and flexibly, while allowing the collaboration to effectively share with each other and work from virtually any analysis framework.

In conclusion, this limited amount of data produced by SeaQuest has already increased the precision and constraints on the nuclear modification of sea quarks in the nucleon. With further refinement of these procedures and additional data, SeaQuest will assuredly be able to provide the best, most precise data on the A-dependence of the Drell-Yan process.