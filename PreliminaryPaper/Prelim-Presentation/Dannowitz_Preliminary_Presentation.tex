\documentclass{beamer}

\mode<presentation>
{
  \usetheme{Warsaw}
  \setbeamercovered{transparent}
  \setbeamertemplate{navigation symbols}{}
}

\usepackage[absolute,overlay]{textpos}
\newenvironment{reference}[2]{%
  \begin{textblock*}{\textwidth}(#1,#2)
      \footnotesize\it\bgroup\color{red!50!black}}{\egroup\end{textblock*}}

\expandafter\def\expandafter\insertshorttitle\expandafter{%
  \insertshorttitle\hfill%
  \insertframenumber}


\usepackage[english]{babel}
\usepackage[utf8]{inputenc}
\usepackage[T1]{fontenc}

\title[The Nuclear EMC Effect]{A Drell-Yan Study of the EMC Effect and Short Range Correlations at SeaQuest}
\subtitle{}

\author[Bryan Dannowitz]
{B.~Dannowitz \newline Advisor: Naomi C.R. Makins}

\institute{SeaQuest at Fermi National Accelerator Laboratory \\ The University of Illinois at Urbana-Champaign}

\date{October 19th, 2011}

%\pgfdeclareimage[height=1.0cm]{UIUCLogo}{UIUC_Logo.jpg}
%\logo{\pgfuseimage{UIUCLogo}}

\begin{document}

\begin{frame}
  \titlepage
\end{frame}

\begin{frame}{Outline}
  \tableofcontents
\end{frame}

\section{Introduction: SeaQuest, Nucleons, and The Drell-Yan Process}

\begin{frame}{The SeaQuest Experiment}{Spectrometer}
  \begin{columns}

\column{.5\textwidth}
	\begin{itemize}
	\item
	Solid Iron Magnet
	\item
	Open-air Spectrometer Magnet
	\end{itemize}
\column{.5\textwidth}
	\begin{itemize}
	\item
	Hodoscopes for triggering
	\item
	Drift Chambers for tracking
	\end{itemize}
\end{columns}
	
\end{frame}


\end{document}


