\chapter{PMT Upgrade}

\begin{figure}
	\centerline{
		\mbox{\includegraphics[width=0.5\textwidth]{figures/nosag.jpg} \includegraphics[width=0.5\textwidth]{figures/sag.jpg}}
	}
	\caption{(Left) Histogram of hodoscope `hits' in a typical event; (Right) Histogram of high-intensity event, with marked sagging most noticeably in the middle of the y-measuring hodoscopes}
	\label{fig:sag}
\end{figure}

During Run I of SeaQuest, observations of hodoscope wire maps (as in Fig.~\ref{fig:sag}) suggested an apparent drop in expected performance in the $y-$measuring hodoscopes. While this performance was most obviously seen in the $y-$measuring hodoscope planes, the $x-$measuring planes were likely also affected. This effect was assumed to be due to high-intensity RF buckets that caused very high multiplicity in all of the detectors in the spectrometer for that event. The result of these intense events seemed to push the PMTs and/or their PMT base electronics past their operational capacity.

The understood cause of this \emph{``sag''} in performance, as it came to be called, was due to a destabilization in the voltage divider in the PMT base. This critical component holds each dynode stage at a specific voltage, and when this destabilizes and is unable to maintain an appropriate voltage difference between dynode stages, inefficient performance of the PMT results.

During the Fall of 2012, prototyping and testing was performed with the goal in mind being to assemble a new base for the Philips XP-2008 PMTs~\cite{tubespecs} and compare its performance to the original PMT base and to some modern, high-performance Hamamatsu PMTs. Once a base design tested well, the new bases would be manufactured and installed in the existing frames of the original PMT bases.

\section{PMT Basic Construction and Operation}

\begin{figure}
	\centering
	\includegraphics[width=0.5\textwidth]{figures/pmt-diagram.png}
	\caption{A diagram of typical PMT operation. The circuit controlling the voltage-dropping resistors is the part that was upgraded in this chapter.}
	\label{fig:pmt}
\end{figure}

Figure~\ref{fig:pmt} shows a schematic design of a typical photomultiplier tube and base setup. It consists of a photocathode that is followed by an electron multiplier section (or dynode string) then an anode from which a final signal is delivered. During operation, a high voltage is applied to the photocathode, dynodes, and anode in such a way that there's a potential ``ladder'' going from stage to stage. When an incident photon from the hodoscope scintillator paddle hits the photocathode, an electron is emitted via the photoelectric effect. The voltage difference between the cathode and dynode stages draws the emitted electron to the dynodes, and each time an electron hits a dynode, some of that electron's energy is transferred to other electrons in the dynode. These electrons then are emitted and become accelerated towards the next dynode stage. This process is called secondary emission, and by the time the process is repeated, there is a cascade or avalanche of electrons that land on the anode, resulting in a signal that can be amplified and analyzed.

It is the case that the voltage divider ultimately supplies the electrons that are emitted in this signal cascade. If too many photons and resultant electron cascades occur, the dynode stages' voltage divider will destabilize as they attempt to resupply the the dynode stages with electrons. The problem that was experienced at SeaQuest was that these high-intensity events were flooding the PMTs with photons, causing this ``saturation'' which caused this destabilization and the inefficient performance that was observed. The goal specifically was to test out modern base designs that provided for added stability to the performance of the voltage divider, even under high rates.

In general, each base divides around a \unit[-1500]{V} potential total over the photocathode (K), ten dynode stages (D1-D10), and the anode (A). There are two currents that are referred to here:
\begin{itemize}
	\item Signal Current: This is the signal that passes over the anode, which is the end-result of the cascading secondary emission electrons from each dynode stage.
	\item Bleeder Current: This is the current through the voltage divider. It is termed the ``bleeder'' current since the compounding electrons in the signal current must be ``bled'' from the current through the voltage divider.
\end{itemize}

Throughout these voltage base designs, capacitors are commonly implemented in the latter dynode stages where the most electrons are emitted. These capacitors, when charged, are able to replenish the lost charge on its corresponding dynode stage in the event that an intense light pulse induces a large signal current.  As the capacitor is able to hold its own charge, this resupply can occur without requiring the charges to be drawn from the bleeder current, thereby keeping the voltage across the dynode stages more stable.

\section{PMT Base Design Iterations}

There were several iterations of base design to determine which was best to approach for full base production and installation at SeaQuest. The core addition was the inclusion of transistors between dynode stages, according to the improvements suggested by C.R. Kerns in his paper regarding high-rate PMT bases~\cite{Kerns:1977qr}. Common solutions to destabilization in PMT bases have been to have (1) very large capacitor banks with charges $> 10^3$ times greater than the time-averaged dynode current and/or (2) miniature on-board,separately-powered Cockroft-Walton power supplies for the final dynode stages. A Kerns-style transistorized base allows for a light-weight, small size, and simple base that does not require extra power supplies or voluminous energy storage capacitors.

In general, there are three important features that were tuned in this set of prototypes that affected the performance of the phototubes:
\begin{itemize}
	\item
	Lower resistance
	\item
	Transistors (with protective diodes)
	\item
	Higher capacitance between dynode stages
	\item
	Distribution of voltage division
\end{itemize}

Lower overall resistance of the voltage divider increases the bleeder current. This means that the base will be more capable of handing high-intensity, as it will be better able to replenish the charges on each dynode stage in the case of a large signal. Typically, the larger the bleeder current, the larger the signal current can be without destabilizing the voltage divider. Higher rates usually put higher demand on the signal current, so by reducing the overall resistance, one can easily increase the rate capability. The shortfall here is that with voltage constant and resistance decreased, according to Ohm's Law ($V = IR$), the current will increase. As a result, the power dissipated by the circuit ($P = I^2 R$) will go as $I^2$, and the PMT base may heat up to critical temperatures faster than it can dissipate the heat as there is no significant ventilation in the PMT base enclosure. The Philips XP-2008 manual quotes that for continuous usage and storage the ambient temperature should not exceed $50^\circ C$ ($122^\circ F$). In addition to heat concerns, there's typically a power rating for the class of small on-board resistors that were planned to be used. Approaching or exceeding that power limit would run the risk of burning out a resistor and rendering the base inoperable. 

\begin{figure}
	\centering
	\includegraphics[width=0.5\textwidth]{figures/MOSFET_Structure.png}
	\caption{MOSFET showing gate (G), body (B), source (S) and drain (D) terminals. The gate is separated from the body by an insulating layer (white)~\cite{wmc:mosfet}}
	\label{fig:mosfet}
\end{figure}

Metal–oxide–semiconductor field-effect transistors (MOSFETs) are introduced here to maintain the proper voltage division. In general, MOSFET transistors have an \emph{source}, a \emph{drain}, and a \emph{gate}, where current flows freely through from the source to the drain, gate permitting. If at any point a certain voltage across the gate of the transistor is not supplied (here, the voltage across dynode stages), then the source-to-drain current through the transistor is stopped until the proper gate voltage is restored. This helps greatly to ``intelligently'' regulate the voltage across the dynodes. Wherever transistors are used, diodes are also implemented to prevent the unlikely case of a current moving across the transistors' gate in the wrong direction. This protects the transistors from being damaged particularly when powering the circuit on and off.

Having capacitors along the higher dynode stages allows for quick resupply of charges to the dynodes, thus maintaining proper voltage division. This, however, is only a stop-gap measure and is only effective under cases of high instantaneous currents. It is the case that, should there be a constant too-high rate of operation, the capacitors will not be able to recharge themselves. Typical recharge time for the capacitors discussed in this section can range from \unit[0.1-1]{ms}.

Finally, the specific division of voltage across each stage, from D1 to A, has an influence on the behavior of the PMT operation. As we see from the operations manual of the phototube in Fig.~\ref{fig:voltage_schemes}, in the case of a progressively increasing voltage division, there is a good compromise between timing and linearity. With respect to phototube operation, ``linearity'' is the quality that the amount of charge deposited on the anode is linearly proportional to the energy of the incident photon. ``Timing'', on the other hand, is the quality that the time it takes for a high-energy photon signal and a low-energy photon signal to progress through the stages should be the same. For the purpose of optimization at SeaQuest, we wish to optimize the amount of signal (i.e. amplification or \emph{gain}) that the phototube can accommodate. For this, the recommended voltage partitioning is flat from D1-A~\cite{tubespecs}.

It should be noted that the prototype iterations of PMT base design changes was not intended to cover the entire phase space of these tunable parameters. The purpose of these tests were to get a sense of how changing one or more of these parameters could affect the PMT high-rate capabilities. The decision for a final base design had to be decided on with relative speed to get them manufactured, built, and installed before Run II of the experiment. As such, true optimization of all parameters could not be achieved within the scope of these tests.

\begin{figure}
	\centering
	\includegraphics[width=0.7\textwidth]{figures/voltage_divider.png}
	\caption{Suggested voltage division schemes for gain vs. timing/linearity compromise~\cite{tubespecs}.}
	\label{fig:voltage_schemes}
\end{figure}

\subsection{Original Base}

The base that came attached to the PMTs were manufactured specifically for use by the ARGUS experiment, which was a relatively (by SeaQuest standards) lower-rate collider experiment that used $e^+ e^-$ annihilation at the \emph{DORIS\ II} ring at DESY. After their tenure at ARGUS, they were handed down to the HERMES experiment located at the \emph{HERA} polarized electron accelerator at DESY.

Though no actual circuit diagram was documented for the original PMT base, it was dissected and each component was measured. The results can be found in Fig.~\ref{fig:original-board}, and its voltage division seen in Fig.~\ref{fig:original-volt}. It features a simple string of resistors with capacitors of increasing capacitance along the last six stages. The voltage division can be recognized to be similar to the timing-linearity compromise scheme described in Fig.~\ref{fig:voltage_schemes}. With a total resistance of approximately \unit[3.95]{$M\Omega$} and the operational voltage of \unit[-1500]{V}, the expected standing (bleeder) current even when sitting in the dark is expected to be \unit[0.38]{mA}. With this in mind, we would expect the voltage divider to destabilize when the signal current approaches this value.

\begin{figure}
	\centerline{
		\mbox{\includegraphics[width=0.75\textwidth]{figures/pmt.png}}
	}
	\caption{The original PMT base inherited from the ARGUS and HERMES experiments.}
	\label{fig:original-board}
\end{figure}

\begin{figure}
	\centerline{
		\mbox{\includegraphics[width=0.55\textwidth]{figures/original-volt.jpg}}
	}
	\caption{The voltage division between subsequent stages for the original PMT base design when supplied with \unit[-1500]{V}.}
	\label{fig:original-volt}
\end{figure}

\subsection{Prototype Base v1}

Once the task was set to update the PMT base design, the Fermilab Particle Physics Division was consulted on the matter. In 2010 a base design with similar goals for the exact same PMT model was designed by Sten Hansen~\cite{pc:sten}. The circuit diagram for the new base can be found in Fig.~\ref{fig:v1-board}.

Here, the resistance was significantly reduced by a factor of about 2.9, allowing for much more bleeder current, without exceeding or closely approaching the on-board resistors' power rating. Also, the voltage division was designed to be relatively ``flat'' (Fig.~\ref{fig:v123-volt}) across stages from D1 to A, which is stated to be recommended for optimal gain. With a total resistance of approximately \unit[1365]{$k\Omega$} and the operational voltage of \unit[-1500]{V}, the expected standing (bleeder) current even when sitting in the dark is expected to be \unit[1.1]{mA}. Already here, we see that this design parameter alone suggests its ability to withstand $\sim 3x$ more signal current as compared to the original base.

The introduction of MOSFET transistors is seen between each stage from D7 to D10. Sending the current in parallel over a \unit[1]{$M\Omega$} resistors allows the gate of the transistor to measure the voltage without drawing much current. The Zener dynodes are there in place before each transistor gate to ensure that current only goes one way across the sensitive gate channels.

It is also notable that there are banks of capacitors in parallel across the higher dynode stages. Since there is higher current through this circuit than the original base under the same voltage, there will be greater demand on the capacitors to resupply the dynode stages with spent charge. The two \unit[10]{nF} capacitors in parallel across each stage (which amount to \unit[20]{nF} total) is significantly greater than the capacitance across the stages of the original base.

\begin{figure}
	\centerline{
		\mbox{\includegraphics[width=0.7\textwidth]{figures/newbase.png}}
	}
	\caption{The Prototype v1 board circuit diagram received from Fermilab Particle Physics Division~\cite{pc:sten}. The parts are denoted as R: resistor, C: capacitor, Q: MOSFET transistor, D: Zener diode.}
	\label{fig:v1-board}
\end{figure}

\begin{figure}
	\centerline{
		\mbox{\includegraphics[width=0.55\textwidth]{figures/v123-volt.jpg}}
	}
	\caption{The voltage devision between subsequent stages for the Prototypes v1, v2, and v3 PMT base designs  when supplied with \unit[-1500]{V}.}
	\label{fig:v123-volt}
\end{figure}

\subsection{Prototype Base v2}

The first modification made to the prototype board was to keep everything identical except for the total resistance of the circuit. This was accomplished by halving the resistance of each of the first six stages (R6-R13 on Fig.~\ref{fig:v1-board}) from to increase the bleeder current. The resulting current of the base at \unit[-1500]{V} is at around \unit[2.2]{mA}. The voltage division retained the same values as described in Fig.~\ref{fig:v123-volt}.

\subsection{Prototype Base v3}

In the case that destabilization was occurring prior to the dynode stages with the added capacitors and transistors, the third prototype was decided to take the Prototype v1 design and add more ``transistorized'' stages earlier on. This entailed extending the parallel configuration of capacitors, transistor, diode, and \unit[1]{$M\Omega$} to the D5-D6 and D6-D7 stages. This prototype configuration can be seen in Fig.~\ref{fig:v3-board}). The voltage division was not significantly altered by this change and remained relatively the same as what is described in Fig.~\ref{fig:v123-volt}.

\begin{figure}
	\centerline{
		\mbox{\includegraphics[width=0.7\textwidth]{figures/newbase_6mosfet.png}}
	}
	\caption{The Prototype v3 board: 3 more transistorized stages than the Prototype v1 design.}
	\label{fig:v3-board}
\end{figure}

\subsection{Prototype Base v4}

The final modification arose from a suggestion from a Fermilab collaborator. It was suggested that it may significantly extend the dynamic range of the tube/base by increasing the voltage drop in, specifically, the last stage relative to the other stages.  This reduced to simply replacing R5 (of Fig.~\ref{fig:v1-board}), a \unit[1]{$M\Omega$} resistor, with a \unit[1.5]{$M\Omega$}.  All of the rest would remain unchanged from Prototype v1. The premise for this modification was in the case that the final batch of electrons needed help being ``swept'' to the anode with a higher voltage difference. The change applied resulted in the voltage distribution accorging to Fig.~\ref{fig:v4-volt}.

\begin{figure}
	\centerline{
		\mbox{\includegraphics[width=0.55\textwidth]{figures/v4-volt.jpg}}
	}
	\caption{The (negative) voltage between subsequent stages for the Prototype v4 PMT base.}
	\label{fig:v4-volt}
\end{figure}

\section{PMT Base Comparisons}

There is a specific difficulty with the objective to increase the rate capability of our PMT's. This difficulty is that there was not a known instantaneous intensity or target rate capability to attain. The rate and intensity that caused the original PMT performance to sag is unknown, and if it was known, it would be difficult to match the intensity with an experimental setup. As a result, the objective in these tests was to \emph{compare} the performance of the same PMT under controlled conditions using the original and various prototype bases.

Due to the effects of using different PMTs and due to temperature and humidity fluctuations, the PMT behavior can be somewhat variable from test to test. For this reason, one can only reasonably compare results within each base comparison test, and not across different comparison tests. Each was performed on different days, and sometimes with different PMT’s.

\subsection{Testing Apparatus and Measurments}

In this experiment, a PMT attached to a PMT base was placed into a light-tight box facing a fast-pulsing 470nm wavelength LED, which provides the driving photonic signal. The LED itself was driven by an Agilent 33520 Function / Arbitrary Waveform Generator capable of generating signals up to 30MHz. The LED's unadulterated intensity was by far too much light to perform any useful test with such photosensitive hardware. Its intensity was attenuated by use of a neutral density filter (NDF), with a rating $D=3.0$, where the NDF allows 1 in $10^D$ photons through (1 in 1000 for $D=3.0$).

These were all kept within the light-tight box\footnote{These light-tight boxes are paradoxically referred to as both ``light boxes'' and ``dark boxes'', as they're used for testing \emph{light}-sensitive equipment and they're made to be \emph{dark} inside.} loaned to the SeaQuest collaboration from the Daya Bay experiment. This light box was equipped with a patch panel that had both BNC and SHV connectors by which to power the PMT base and read out its signal.

A simple data acquisition was assembled from an amplifier, discriminator, and scaler in order to observe that the PMT was functioning properly and firing off at the rate that the LED was set to pulse at. The PMT base was powered by a high voltage supply, and an ammeter was connected between the two in order to measure the amount of current drawn, or bleeder current, from the HV power supply. The PMT signal was processed by an oscilloscope, averaging the pulse over 300 pulses. The primary measurement was measuring the area of the averaged pulse ($V\cdot s$).

We calculated the signal current from the anode as:
\begin{eqnarray}
	Q_{pulse} & = & \frac{\int V dt}{R} \\
	I_{signal} & = & f Q_{pulse}
\end{eqnarray}
where $f$ here is the driving frequency of the pulsing LED, $R$ is the termination resistance of the signal (\unit[50]{$\Omega$}), and $\int V dt$ is the integrated area of the averaged PMT pulse. The amplitude of the pulses was also measured, as it was important in determining if it were feasible to remove the typically noisy hodoscope amplifiers from the SeaQuest stations 1 and 2 DAQ setup.

The measured and calculated quantities of interest are:
\begin{itemize}
	\item HV supply (bleeder) current
	\item Averaged signal amplitude
	\item Averaged signal area
	\item Signal current over the anode
\end{itemize}

\begin{figure}
	\centerline{
		\mbox{\includegraphics[width=0.75\textwidth]{figures/setup.jpg}}
	}
	\caption{Inside of a lightbox, we have our prototype board (left) wired up to a Philips XP-2008 PMT (middle), facing a fast-led source (right) }
	\label{fig:setup}
\end{figure}
